\documentclass[12pt,a4paper]{article}                                                                            
\setlength{\parindent}{2em}          % 首行空两字
\usepackage{fontspec}                % 设置字体
\setmainfont{宋体}
\usepackage{indentfirst}             % 首行缩进

%%%%%%%%%% 数学符号公式 %%%%%%%%%%
\usepackage{xeCJK}                   % 中英文混排
\usepackage{amsmath}                 % AMS LaTeX宏包
%\usepackage{amssymb}                 % 用来排版漂亮的数学公式
%\usepackage{amsbsy}
\usepackage{amsthm}
\usepackage{amsfonts}
\usepackage{mathrsfs}                % 英文花体字体
\usepackage{bm}                      % 数学公式中的黑斜体
\usepackage{bbding,manfnt}           % 一些图标,如 \dbend
\usepackage{lettrine}                % 首字下沉,命令\lettrine
\def\attention{\lettrine[lines=2,lraise=0,nindent=0em]{\large\textdbend\hspace{1mm}}{}}
%\usepackage{relsize}                 % 调整公式字体大小:\mathsmaller,\mathlarger
%\usepackage{caption2}                % 浮动图形和表格标题样式

%%%%%%%%%% 图形支持宏包 %%%%%%%%%%
\usepackage{graphicx}                % 嵌入png图像
\usepackage{color,xcolor}            % 支持彩色文本、底色、文本框等
\usepackage{subfigure}
\usepackage{epsfig}                 % 支持eps图像
%\usepackage{picinpar}               % 图表和文字混排宏包
%\usepackage[verbose]{wrapfig}       % 图表和文字混排宏包
%\usepackage{eso-pic}                % 向文档的部分页加n副图形, 可实现水印效果
%\usepackage{eepic}                  % 扩展的绘图支持
%\usepackage{curves}                 % 绘制复杂曲线
%\usepackage{texdraw}                % 增强的绘图工具
%\usepackage{treedoc}                % 树形图绘制
%\usepackage{pictex}                 % 可以画任意的图形
%\usepackage{hyperref}

%%%%%%%%%%% 列表 %%%%%%%%%%%%%
\usepackage{enumerate}

%%%%%%%%%% 正文 %%%%%%%%%%
\begin{document}
%%%%%%%%%% 定理类环境的定义 %%%%%%%%%%
%% 必须在导入中文环境之后
\newtheorem{example}{例}             % 整体编号
\newtheorem{algorithm}{算法}
\newtheorem{theorem}{定理}[section]  % 按 section 编号
\newtheorem{definition}{定义}
\newtheorem{axiom}{公理}
\newtheorem{property}{性质}
\newtheorem{proposition}{命题}
\newtheorem{lemma}{引理}
\newtheorem{corollary}{推论}
\newtheorem{remark}{注解}
\newtheorem{condition}{条件}
\newtheorem{conclusion}{结论}
\newtheorem{assumption}{假设}

%%%%%%%%%% 一些重定义 %%%%%%%%%%
\renewcommand{\contentsname}{目录}     % 将Contents改为目录
\renewcommand{\abstractname}{摘要}     % 将Abstract改为摘要
\renewcommand{\refname}{参考文献}      % 将References改为参考文献
\renewcommand{\indexname}{索引}
\renewcommand{\figurename}{图}
\renewcommand{\tablename}{表}
\renewcommand{\appendixname}{附录}
\renewcommand{\proofname}{证明}
\renewcommand{\algorithm}{算法}

%%%%%%%%%% 论文标题、作者等 %%%%%%%%%%
\title{基于p模的协同表达高光谱分类模型}
\author{雷开宇,中国农业大学数学与应用数学系}
%\date{2008年10月01日}                % 日期
\date{2018年4月7日}
\maketitle                            % 生成标题
\tableofcontents                      % 插入目录
\thispagestyle{empty}                 % 首页无页眉页脚

\begin{abstract}
\noindent TODO: 补充摘要,中文摘要300字左右,英文摘要250词左右
\end{abstract}

%\PencilRightUp % 一些可爱的图标,需要bbding宏包的支持
公元1974年,ACM图灵奖授予了Standford大学教授\index{Donald E. Knuth} Donald E. Knuth (高德纳),
表彰他在算法和程序语言设计等多方面杰出的成就。他的巨著 The Art of Computer Programming 令人震撼。
另外,Knuth的突出贡献还包括\index{\TeX 系统} \TeX 系统,毫不夸张地评价,\TeX 给排版带来了一场革命。




%%%%%%%%%% section:选题背景 %%%%%%%%%%
\section{绪论}
\indent   % 恢复缩进

在信息时代....

\subsection{研究背景及意义}
高光谱图像(Hyperspectral Imagery, HSI),是一种通过遥感技术获得的图像。高光谱影像收集及处理整个跨电磁波谱的信息,这些信息是由高光谱传感器(Hyperspectral image sensors)获取并整理。正如人眼一样,人眼能够看到物体,是因为物体反射的光,或者自身发出的光进入眼睛,大脑感受到信号,生成物体影响。与人眼不同的是,人眼只能分辨出可见光,而高光谱图像则可以延伸至红外、紫外,甚至整个电磁波谱上。\\

HSI被广泛应用于农业、地质、天文、化学、食品工程和环境工程等领域,在军事上也有着重要的作用。研究者通过分析HSI来达到不同的目的。例如在地质勘测中,可以通过遥感获取的HSI图像来定位矿区的主要矿种和位置,在环境工程中,通过对有毒气体泄漏区域的空气进行HSI分析,可以确定毒气成分和扩散趋势。

HSI分析中,分类问题是研究的一个热门。在现实中,人们往往能通过专业领域知识来确定出HSI中某个区域的物质构成,以及少量的分布信息。例如在一个利用HSI进行农产品种植结构分析的例子中,人们往往可以通过各地农业部门上报的信息,来确定HSI中存在几种农作物,并且可以根据采样调查知道某个区域农作物的具体分布情况。如何通过这些有限的信息来对整个HSI中的物质进行分类,这是一个值得研究的问题。

机器学习在HSI分类中有着广泛的应用。当人们获取到HSI时,由于人类自身生理结构和极限的限制,直接对HSI进行分析无疑需要大量的人力物力,成本过于高昂。同时,人工方式分类HSI还会受到人类主观心理的影响,例如对图像中特定区域,不同的人可能认为该区域的成分分布不同。随之HSI传感器分辨率越来越高,通过人力直接进行分类变得不再可行,因此使用计算机来对HSI进行分析变得越来越重要。在分类问题中,机器学习能够自主地通过已知信息来学习,因此十分适合HSI分类问题。

从样本数量来说,机器学习分为三类:监督学习、半监督学习和无监督学习。监督学习是将已知标签的样本作为训练集,对样本特征进行学习。无监督学习是指作为训练集的样本没有标签信息,典型的无监督学习有聚类问题。半监督学习是指训练集中既有有标签的样本点,又有无标签的样本点。在本文的假定下,HSI分类问题是一个半监督学习问题,通过数量较少的有标签样本点和部分无标签的样本点构造算法,实现对HSI图像的自动分类。

\subsection{研究现状}
半监督学习是利用....

\subsection{本文主要工作}
本文主要针对HSI问题的,对协同表达算法进行了改进,使之在HSI分类中表现更好。主要工作有以下几点:
\begin{enumerate}[(1)]
\item 在CR模型的基础上,引入了p模技术,把正则项的中的范数改为p范数,取得更好的稀疏性。
\item 在JCR模型的基础上,引入了p模技术,以得到更好的稀疏性和精确度。
\item 将本文改进后的p-CRC模型、p-JRC模型分别在Indian、Pavia、Salinas等数据集上,与SRC、CRC、JCR和SVM方法进行比较,分析实验结果。
\end{enumerate}

\subsection{本文组织结构}







%%%%%%%%%%% section:方案论证 %%%%%%%%%%
\section{方案论证}
这部分主要介绍协同表达模型特点,以及在HSI领域的应用情况


\begin{theorem}[L\'{e}vy\index{L\'{e}vy 定理}]
令 $F(x),\varphi(t)$ 分别为随机变量 $X$ 的分布函数和特征函数。
假定 $F(x)$ 在 $a+h$ 和 $a-h (h>0)$ 处连续,则有
\begin{align}
 \label{Levy theorem}  % 方程的标记可以是专有名词
F(a+h)-F(a-h)&=\lim_{T\rightarrow\infty}\frac{1}{\pi}\int^{T}_{-T}\frac{\sin ht}{t} 
e^{-ita}\varphi(t)dt
\end{align}
\end{theorem}
\begin{proof}
  从略。感兴趣的读者可以参考……。
\end{proof}


\begin{corollary}
密度函数和特征函数之间有如下的关系。
\begin{align}
 \label{DensityCharacteristic}   % 自定义的标记
  f(x)&=\frac{1}{2\pi}\int^{+\infty}_{-\infty} e^{-itx}\varphi(t)dt
\end{align}
\end{corollary}

\begin{proof}
由公式 (\ref{Levy theorem}) 和 Lebesgue 定理,我们有
\begin{align*}
 \frac{F(x+\Delta x)-F(x)}{\Delta x}&=\frac{1}{2\pi}\int^{+\infty}_{-\infty}
 \frac{\sin(t\Delta x/2)}{t\Delta x/2} e^{-it(x+\Delta x/2)}\varphi(t) dt\\
  f(x)&=\frac{1}{2\pi}\int^{+\infty}_{-\infty}\lim_{\Delta x\rightarrow 0}
 \frac{\sin(t\Delta x/2)}{t\Delta x/2} e^{-it(x+\Delta x/2)}\varphi(t) dt\\
  &=\frac{1}{2\pi}\int^{+\infty}_{-\infty} e^{-itx}\varphi(t)dt\qedhere
\end{align*}
\end{proof}

我们知道特征函数的定义是

\begin{align}
 \label{section1:characteristic}   % 标记中注明了章节号
 \varphi(t)&= E(e^{itX})=\int^{+\infty}_{-\infty} e^{itx} f(x)dx
\end{align}

L\'{e}vy 定理在分布函数和特征函数之间搭建了一座桥梁。
对比 (\ref{DensityCharacteristic}) 和 (\ref{section1:characteristic}) 可见,
密度函数和特征函数之间的关系非常巧妙。

                                                                                                                 
\attention 在\TeX 环境里,数学公式的表达是很自然的,绝大多数命令就是英文的数学专有名词或它们的缩写,
如果你以前读过英文的数学文献,记忆这些命令是不难的。如果你没读过,正好通过记忆这些命令来了解术语。


手头有个命令快速寻查表是很方便的,我用的是 Hypertext Help with \LaTeX,网上可以搜到,是免费的。

\index{贝叶斯方法}贝叶斯方法\cite{Gelman} 主要用于小样本数据分析,它利用参数先验分布和
后验分布之差异进行统计推断,其一般步骤是:

\begin{enumerate}
 \item 构建概率模型,包括参数的先验分布。
 \item 给定观察数据,计算参数的后验分布。
 \item 分析模型的效果,如有必要,回到第一步。
\end{enumerate}

\begin{example}
下面,我们给一个表格的例子,一个图形的例子。

\begin{center}
\begin{table}[!ht]     % 强制在原位显示表格
\centering
\caption{二维随机向量$(X,Y)$的边缘分布}
\begin{tabular}{l|ccccc|c}
  $_X$\hspace{3mm} $^Y$&$y_1$&$y_2$&$\cdots$&$y_j$&$\cdots$\\
\hline
$x_1$   &$p_{11}$&$p_{12}$&$\cdots$&$p_{1j}$&$\cdots$&$p_{1\cdot}$\\
$x_2$   &$p_{21}$&$p_{22}$&$\cdots$&$p_{2j}$&$\cdots$&$p_{2\cdot}$\\
$\vdots$&$\vdots$&$\vdots$&$\vdots$&$\vdots$&$\vdots$&$\vdots$\\
$x_i$   &$p_{i1}$&$p_{i2}$&$\cdots$&$p_{ij}$&$\cdots$&$p_{i\cdot}$\\
$\vdots$&$\vdots$&$\vdots$&$\vdots$&$\vdots$&$\vdots$&$\vdots$\\
\hline
   &$p_{\cdot 1}$&$p_{\cdot 2}$&$\cdots$&$p_{\cdot j}$&$\cdots$&1
\label{marginal distribution}
\end{tabular}
\end{table}
\end{center}

在表\ref{marginal distribution} 中,$p_{\cdot j}=\sum\limits_i p_{ij}$,
类似地,$ p_{i\cdot}=\sum\limits_j p_{ij}$。
\end{example}

% 插入一个图片





%%%%%%%%%%% section:过程论述 %%%%%%%%%%
\section{过程论述}
这个部分包括模型改进介绍,包括已有模型、改进后的模型、改进后的算法,改进后的实现




%%%%%%%%%%% section:实验与分析 %%%%%%%%%%
\section{实验与分析}
这个\TeX 模板只是为了提供一个学习\TeX 的参考,各节的内容并没有关联性。欢迎读者使用并改进该模板,
并祝学习\TeX 愉快!

Knuth大师最初设计\TeX 的时候并没有想到中文化,\TeX 排版系统的中文化始终令初学者望而却步、云山雾罩。
类UNIX系统下的teTeX和Windows系统下的MikTeX,都是\TeX 知名的发行版。然而,teTeX已经停止研发五年之久,
基于MikTeX的中文发行版CTeX 虽然如火如荼,但依然挡不住\TeX Live 一统江湖的大趋势。

虽然\TeX Live 还未入住FreeBSD的ports tree,但teTeX的远去,
令FreeBSD之下的很多ports不得不面临改换门庭的窘境。例如,auctex、latex-cjk等等。


\TeX 的中文化可以有多种途径,xelatex 是其中最简单的(不见得是最美观的)。
在\TeX Live 2011 之下,不需要有任何更多的设置,甚至不用考虑中英文混排,
xelatex能满足绝大多数中文化要求。这对于初学者来说,无疑是一个福音。


%%%%%%%%%% 参考文献 %%%%%%%%%%
\begin{thebibliography}{}
\bibitem[Gelman et al., 2004]{Gelman} Gelman, A., Carlin, J. B., Stern, H. S. \& Rubin, D. B.
 (2004) Bayesian Data Analysis (Second Edition). \newblock Chapman \& Hall/CRC.
\end{thebibliography}
\clearpage
\end{document}
%%%%%%%%%% 结束 %%%%%%%%%%